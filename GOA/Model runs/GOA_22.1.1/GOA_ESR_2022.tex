% Options for packages loaded elsewhere
\PassOptionsToPackage{unicode}{hyperref}
\PassOptionsToPackage{hyphens}{url}
%
\documentclass[
]{article}
\usepackage{amsmath,amssymb}
\usepackage{lmodern}
\usepackage{iftex}
\ifPDFTeX
  \usepackage[T1]{fontenc}
  \usepackage[utf8]{inputenc}
  \usepackage{textcomp} % provide euro and other symbols
\else % if luatex or xetex
  \usepackage{unicode-math}
  \defaultfontfeatures{Scale=MatchLowercase}
  \defaultfontfeatures[\rmfamily]{Ligatures=TeX,Scale=1}
\fi
% Use upquote if available, for straight quotes in verbatim environments
\IfFileExists{upquote.sty}{\usepackage{upquote}}{}
\IfFileExists{microtype.sty}{% use microtype if available
  \usepackage[]{microtype}
  \UseMicrotypeSet[protrusion]{basicmath} % disable protrusion for tt fonts
}{}
\makeatletter
\@ifundefined{KOMAClassName}{% if non-KOMA class
  \IfFileExists{parskip.sty}{%
    \usepackage{parskip}
  }{% else
    \setlength{\parindent}{0pt}
    \setlength{\parskip}{6pt plus 2pt minus 1pt}}
}{% if KOMA class
  \KOMAoptions{parskip=half}}
\makeatother
\usepackage{xcolor}
\usepackage[margin=1in]{geometry}
\usepackage{graphicx}
\makeatletter
\def\maxwidth{\ifdim\Gin@nat@width>\linewidth\linewidth\else\Gin@nat@width\fi}
\def\maxheight{\ifdim\Gin@nat@height>\textheight\textheight\else\Gin@nat@height\fi}
\makeatother
% Scale images if necessary, so that they will not overflow the page
% margins by default, and it is still possible to overwrite the defaults
% using explicit options in \includegraphics[width, height, ...]{}
\setkeys{Gin}{width=\maxwidth,height=\maxheight,keepaspectratio}
% Set default figure placement to htbp
\makeatletter
\def\fps@figure{htbp}
\makeatother
\setlength{\emergencystretch}{3em} % prevent overfull lines
\providecommand{\tightlist}{%
  \setlength{\itemsep}{0pt}\setlength{\parskip}{0pt}}
\setcounter{secnumdepth}{-\maxdimen} % remove section numbering
\ifLuaTeX
  \usepackage{selnolig}  % disable illegal ligatures
\fi
\IfFileExists{bookmark.sty}{\usepackage{bookmark}}{\usepackage{hyperref}}
\IfFileExists{xurl.sty}{\usepackage{xurl}}{} % add URL line breaks if available
\urlstyle{same} % disable monospaced font for URLs
\hypersetup{
  hidelinks,
  pdfcreator={LaTeX via pandoc}}

\author{}
\date{\vspace{-2.5em}}

\begin{document}

\hypertarget{multispecies-model-estimates-of-time-varying-natural-mortality-in-the-goa}{%
\subsection{Multispecies model estimates of time-varying natural
mortality in the
GOA}\label{multispecies-model-estimates-of-time-varying-natural-mortality-in-the-goa}}

\emph{Grant Adams\(^1\), Kirstin K. Holsman\(^{1,2}\), Steve
Barbeaux\(^2\), Martin Dorn\(^2\), Pete Hulson\(^3\), Jim Ianelli\(^2\),
Cole Monnahan\(^2\), Kalei Shotwell\(^2\), Ingrid Spies\(^2\), Ian
Stewart\(^4\), and Andre Punt\(^1\)}

\href{mailto:adamsgd@uw.gov}{\nolinkurl{adamsgd@uw.gov}}

\(^1\)School of Aquatic and Fishery Sciences, University of Washington,
Seattle, WA, USA

\(^2\)Resource Ecology and Fisheries Management Division, Alaska
Fisheries Science Center, Seattle, WA, USA

\(^3\)Auke Bay Laboratories, Alaska Fisheries Science Center, Juneau,
AK, USA

\(^4\)International Pacific Halibut Commission, Seattle, WA, USA.

\textbf{Last Updated: September 2022}

\hypertarget{summary-statement}{%
\subsection{Summary statement:}\label{summary-statement}}

The climate-enhanced multispecies model (CEATTLE) for the Gulf of Alaska
(GOA) estimates that natural mortality for age-1 pollock, Pacific cod,
and arrowtooth flounder due to all sources has declined in recent years
and is below the long-term mean. Alternatively, estimates of biomass
consumed of pollock, Pacific cod, and arrowtooth flounder as prey across
all ages has increased and is currently above the long term mean for
pollock and Pacific cod.

\hypertarget{status-and-trends}{%
\subsection{Status and trends:}\label{status-and-trends}}

Estimated age-1 natural mortality (M) for walleye Pollock, Pacific cod,
and arrowtooth flounder peaked in 2005 for pollock, 2005 for P. cod, and
1991 for arrowtooth flounder (Fig. 1). At an average of 1.17
yr\(^{-1}\), age-1 M estimated by CEATTLE was greatest for pollock and
lower for Pacific cod (0.83 yr\(^{-1}\)) and arrowtooth (0.34
yr\(^{-1}\) for females and 0.44 yr\(^{-1}\) for males). After
decreasing in recent years, pollock age-1 M remained lower in 2022 at
1.05 yr\(^{-1}\) (SD = 0.13) relative to the long-term mean 1.17
yr\(^{-1}\) and the values used for single species assessment (age-1 M =
1.39; Fig. 1). Additionally, Pacific cod and arrowtooth flounder age-1 M
were below the long-term mean after decreasing in recent years (Fig. 1),
but above the values used/estimated for the single species assessment of
0.50 yr\(^{-1}\) (Pacific cod), 0.2 yr\(^{-1}\) (arrowtooth females),
and 0.35 yr\(^{-1}\) (arrowtooth males), with total age-1 M at around
0.76 yr\(^{-1}\) (SD = 0.08) for P. cod 0.33 yr\(^{-1}\) (SD = 0.01) for
arrowtooth females, and 0.43 yr\(^{-1}\) (SD = 0.02) for arrowtooth
males. 2022 age-1 M across species is 5.69\% to 30.31\% lower than in
peak years.

On average 151,157 mt of age-1 pollock, 2,484 mt of age-1 Pacific cod,
and 5,432 mt of age-1 arrowtooth flounder was consumed annually by
species included in CEATTLE. Across all ages 589,052 mt of pollock,
29,019 mt of arrowtooth flounder, 5,896 mt of Pacific cod was consumed
annually, on average, by species included in the model. The total
biomass consumed of pollock as prey across all ages increased in 2022
compared to 2021 (Fig. 2). The total biomass consumed of arrowtooth
flounder and Pacific cod has also increased in recent years. The total
biomass consumed of pollock and Pacific cod as prey across all ages is
currently above the long term mean.

\hypertarget{factors-influencing-observed-trends}{%
\subsection{Factors influencing observed
trends}\label{factors-influencing-observed-trends}}

Temporal patterns in total natural mortality reflect annually varying
changes in predation mortality by pollock, P. cod, Pacific halibut, and
arrowtooth flounder that primarily impact age-1 fish (but also impact
older age classes). Predation mortality at age-1 for all species in the
model was primarily driven by arrowtooth flounder (Fig. 3) and
arrowtooth flounder biomass has declined and remained relatively
constant in recent years. Increases in biomass consumed of walleye
Pollock in 2021 relative to 2020 reflect elevated recruitment of age-1
pollock in 2021 that was available to the modelled predators. Combined
annual predation demand (annual ration) of age-4+ pollock, P. cod, and
arrowtooth flounder in 2022 was 6.48 hundred thousand tons, down
slightly from the 7.12 hundred thousand ton annual average (Fig. 4).

\hypertarget{implications}{%
\subsection{Implications:}\label{implications}}

We find evidence of continued decline in predation mortality on age-1
pollock, Pacific cod and arrowtooth flounder due to the species modelled
in CEATTLE. Previous ecosystem modelling efforts have estimated that
mortality of pollock is primarily driven by P. cod (16\%), Pacific
halibut (23\%) and arrowtooth flounder (33\%)(Gaichas et al., 2015).
Declines in total predator biomass are contributing to an overall
decline in total consumption and therefore reduced predation mortality.
Between 1990 and 2010, relatively high natural mortality rates reflect
patterns in annual demand for prey from arrowtooth flounder, whose
biomass peaked during this time period. A strong recruitment of age-1
pollock in 2021 has led to an increase in biomass of pollock being
consumed by predators. Decreases in predation mortality in recent years
suggest that the disappearance of the large age-1 recruitment of pollock
in 2019 was not due wholly to predation by species included in the
model.

\hypertarget{description-of-index}{%
\subsection{Description of index:}\label{description-of-index}}

We report trends in age-1 natural mortality for walleye pollock
(\emph{Gadus chalcogrammus}), P. cod (\emph{Gadus macrocephalus}) and
arrowtooth flounder (\emph{Atheresthes stomias}), from the Gulf of
Alaska (USA). Total natural mortality rates are based on model estimated
sex-specific, time- and age-invariant residual mortality (M1) and model
estimates of time- and age-varying predation mortality (M2) produced
from the multi-species statistical catch-at-age assessment model (known
as CEATTLE; Climate-Enhanced, Age-based model with Temperature-specific
Trophic Linkages and Energetics). The model is based, in part, on the
parameterization and data used for recent stock assessment models of
each species (see Adams et al., 2022 for more detail). The model is fit
to data from five fisheries and seven surveys between 1977 and 2022 and
includes inputs of abundance-at-age from recent stock assessment models
for Pacific halibut scaled to the proportion of age-5+ biomass in IPHC
management area 3 (Stewart \& Hicks, 2021). Model estimates of predation
mortality are empirically derived by bioenergetics-based consumption
information and diet data from the GOA to inform predator-prey
suitability (Holsman \& Aydin, 2015; Holsman, Aydin, Sullivan, Hurst, \&
Kruse, 2019).

\hypertarget{literature-cited}{%
\subsection{Literature Cited}\label{literature-cited}}

Adams, G. D., Holsman, K. K., Barbeaux, S. J., Dorn, M. W., Ianelli, J.
N., Spies, I., Stewart, I. J., et al.~2022. An ensemble approach to
understand predation mortality for groundfish in the Gulf of Alaska.
Fisheries Research, 251: 106303.

Holsman, K. K., Ianelli, J., Aydin, K., Punt, A. E., and Moffitt, E. A.
2016. A comparison of fisheries biological reference points estimated
from temperature-specific multi-species and single-species
climate-enhanced stock assessment models. Deep Sea Research Part II:
Topical Studies in Oceanography, 134: 360--378.

Holsman, KK and K Aydin. (2015). Comparative methods for evaluating
climate change impacts on the foraging ecology of Alaskan groundfish.
Mar Ecol Prog Ser 521:217-23510.3354/ meps11102

Holsman, K.K., Aydin, K., Sullivan, J., Hurst, T., Kruse, G.H., 2019.
Climate effects and bottom-up controls on growth and size-at-age of
Pacific halibut (Hippoglossus stenolepis) in Alaska (USA). Fisheries
Oceanography, 28: 345--358. \url{doi:10.1111/fog.12416}

Gaichas, S., Aydin, K., and Francis, R. C. 2015. Wasp waist or beer
belly? Modeling food web structure and energetic control in Alaskan
marine ecosystems, with implications for fishing and environmental
forcing. Progress in Oceanography, 138: 1--17. Elsevier
Ltd.~\url{http://dx.doi.org/10.1016/j.pocean.2015.09.010}.

Stewart, I., Hicks, A., 2019. Assessment of the Pacific halibut
(\emph{Hippoglossus stenolepis}) stock at the end of 2018. International
Pacific Halibut Commission. Seattle, Wa, USA.

\newpage

\hypertarget{figures}{%
\subsection{Figures:}\label{figures}}

\begin{figure}
\centering
\includegraphics{Results/ESR_Fig1.jpg}
\caption{Annual variation in total mortality (\textbf{M1+M2}) of age-1
pollock (a), P. cod (b), and arrowtooth flounder (females and males) (c)
from the single-species models (dashed line), and the multi-species
models with temperature (points; solid line is a loess polynomial
smoother indicating trends over time). Updated from Adams et al.~2022.}
\end{figure}

\begin{figure}
\centering
\includegraphics{Results/ESR_Fig2.jpg}
\caption{Multispecies estimates of biomass consumed as prey across all
ages by all predators annually in the model of walleye pollock, P. cod,
and arrowtooth flounder. Points represent annual estimates, gray lines
indicate 1979-2022 mean estimates for each species, and the solid line
is a 10 year (symmetric) loess polynomial smoother indicating trends
over time.}
\end{figure}

\begin{figure}
\centering
\includegraphics{Results/ESR_Fig3.jpg}
\caption{Proportion of total predation mortality for age-1 pollock from
pollock (solid), P. cod (dashed), and arrowtooth flounder (dotted)
predators across years. Updated from Adams et al.~2022.}
\end{figure}

\begin{figure}
\centering
\includegraphics{Results/ESR_Fig4.jpg}
\caption{Multispecies estimates of annual ration (hundred thousand tons
consumed per species per year) for adult (age 4 +) predators: a)
pollock, b) P. cod, and c) arrowtooth flounder. Gray lines indicate 1979
-2022 mean estimates and 1 SD for each species; solid line is a 10 y
(symmetric) loess polynomial smoother indicating trends in ration over
time.}
\end{figure}

\end{document}
