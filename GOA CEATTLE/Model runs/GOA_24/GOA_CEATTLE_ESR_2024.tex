% Options for packages loaded elsewhere
\PassOptionsToPackage{unicode}{hyperref}
\PassOptionsToPackage{hyphens}{url}
%
\documentclass[
]{article}
\usepackage{amsmath,amssymb}
\usepackage{iftex}
\ifPDFTeX
  \usepackage[T1]{fontenc}
  \usepackage[utf8]{inputenc}
  \usepackage{textcomp} % provide euro and other symbols
\else % if luatex or xetex
  \usepackage{unicode-math} % this also loads fontspec
  \defaultfontfeatures{Scale=MatchLowercase}
  \defaultfontfeatures[\rmfamily]{Ligatures=TeX,Scale=1}
\fi
\usepackage{lmodern}
\ifPDFTeX\else
  % xetex/luatex font selection
\fi
% Use upquote if available, for straight quotes in verbatim environments
\IfFileExists{upquote.sty}{\usepackage{upquote}}{}
\IfFileExists{microtype.sty}{% use microtype if available
  \usepackage[]{microtype}
  \UseMicrotypeSet[protrusion]{basicmath} % disable protrusion for tt fonts
}{}
\makeatletter
\@ifundefined{KOMAClassName}{% if non-KOMA class
  \IfFileExists{parskip.sty}{%
    \usepackage{parskip}
  }{% else
    \setlength{\parindent}{0pt}
    \setlength{\parskip}{6pt plus 2pt minus 1pt}}
}{% if KOMA class
  \KOMAoptions{parskip=half}}
\makeatother
\usepackage{xcolor}
\usepackage[margin=1in]{geometry}
\usepackage{graphicx}
\makeatletter
\def\maxwidth{\ifdim\Gin@nat@width>\linewidth\linewidth\else\Gin@nat@width\fi}
\def\maxheight{\ifdim\Gin@nat@height>\textheight\textheight\else\Gin@nat@height\fi}
\makeatother
% Scale images if necessary, so that they will not overflow the page
% margins by default, and it is still possible to overwrite the defaults
% using explicit options in \includegraphics[width, height, ...]{}
\setkeys{Gin}{width=\maxwidth,height=\maxheight,keepaspectratio}
% Set default figure placement to htbp
\makeatletter
\def\fps@figure{htbp}
\makeatother
\setlength{\emergencystretch}{3em} % prevent overfull lines
\providecommand{\tightlist}{%
  \setlength{\itemsep}{0pt}\setlength{\parskip}{0pt}}
\setcounter{secnumdepth}{-\maxdimen} % remove section numbering
\ifLuaTeX
  \usepackage{selnolig}  % disable illegal ligatures
\fi
\usepackage{bookmark}
\IfFileExists{xurl.sty}{\usepackage{xurl}}{} % add URL line breaks if available
\urlstyle{same}
\hypersetup{
  hidelinks,
  pdfcreator={LaTeX via pandoc}}

\author{}
\date{\vspace{-2.5em}}

\begin{document}

\subsection{Multispecies model estimates of time-varying natural
mortality in the
GOA}\label{multispecies-model-estimates-of-time-varying-natural-mortality-in-the-goa}

\emph{Grant Adams\(^1\), Kirstin K. Holsman\(^{1,2}\), Pete
Hulson\(^3\), Cole Monnahan\(^1\), Kalei Shotwell\(^1\)}

\href{mailto:grant.adams@noaa.gov}{\nolinkurl{grant.adams@noaa.gov}}

\(^1\)Resource Ecology and Fisheries Management Division, Alaska
Fisheries Science Center, Seattle, WA, USA

\(^2\)School of Aquatic and Fishery Sciences, University of Washington,
Seattle, WA, USA

\(^3\)Auke Bay Laboratories, Alaska Fisheries Science Center, Juneau,
AK, USA

\(^4\)International Pacific Halibut Commission, Seattle, WA, USA.

\subsection{Summary statement:}\label{summary-statement}

The climate-enhanced multispecies model (CEATTLE) for the Gulf of Alaska
(GOA) estimates that natural mortality due to all sources for age-1
pollock and arrowtooth flounder has increased in recent years, but
remain below the long-term mean. However, natural mortality for age-1
Pacific cod has decreased slightly since 2023 and remains below the
long-term mean. Estimates of biomass consumed of pollock, Pacific cod,
and arrowtooth flounder as prey across all ages remains below the long
term mean.

\subsection{Status and trends:}\label{status-and-trends}

Estimated age-1 natural mortality (M) for walleye pollock, Pacific cod,
and arrowtooth flounder peaked in 2003 for pollock, 2005 for Pacific
cod, and 2001 for arrowtooth flounder (Fig. 1). Average age-1 M
estimated by CEATTLE was greatest for pollock (1.5 yr\(^{-1}\)) and
lower for Pacific cod (0.7 yr\(^{-1}\)) and arrowtooth (0.39 yr\(^{-1}\)
for females and 0.48 yr\(^{-1}\) for males). After varying in recent
years, pollock age-1 M increased in 2024 to 1.45 yr\(^{-1}\) and is
currently below the long-term mean of 1.5 yr\(^{-1}\), but above the
value used for single species assessment (age-1 M = 1.39; Fig. 1).
Pacific cod age-1 M decreased slightly to 0.69 yr\(^{-1}\) and remains
below the long-term mean of 0.7 yr\(^{-1}\) (Fig. 1), but above the
age-invariant values estimated in the single species assessment of 0.492
yr\(^{-1}\). Similarly, arrowtooth flounder age-1 M remains below the
long-term mean after increasing slightly in recent years (Fig. 1).
However, arrowtooth age-1 M remains above the values used for the single
species assessment of 0.2 yr\(^{-1}\) (arrowtooth females) and 0.35
yr\(^{-1}\) (arrowtooth males), with total age-1 M at around 0.38
yr\(^{-1}\) for arrowtooth females and 0.48 yr\(^{-1}\) for arrowtooth
males. 2024 age-1 M across species is 4.05\% to 32.79\% lower than in
peak years.

On average 120,379 mt of age-1 pollock, 2,465 mt of age-1 Pacific cod,
and 5,294 mt of age-1 arrowtooth flounder was consumed annually by
species included in CEATTLE between 1977 and 2024. For 2024, we
estimated 50,726 mt of age-1 pollock, 1,763 mt of age-1 Pacific cod,
4,025 mt of age-1 arrowtooth females, and 1,763 mt of age-1 arrowtooth
males was consumed by species included in CEATTLE. Across all ages
394,583 mt of pollock, 25,560 mt of arrowtooth flounder, 4,877 mt of
Pacific cod was consumed annually, on average, by species included in
CEATTLE. The total biomass consumed of pollock and arrowtooth flounder
as prey across all ages increased in 2024 compared to 2023 (Fig. 2). The
total biomass consumed of Pacific cod has decreased in recent years. The
total biomass consumed as prey across all ages for all species is
currently below the long term mean.

\subsection{Factors influencing observed
trends}\label{factors-influencing-observed-trends}

Temporal patterns in total natural mortality reflect annually varying
changes in predation mortality by pollock, Pacific cod, Pacific halibut,
and arrowtooth flounder that primarily impact age-1 fish (but also
impact older age classes). Predation mortality at age-1 for all species
in the model was primarily driven by arrowtooth flounder (Fig. 3) and
arrowtooth flounder biomass has increased in recent years. Combined
annual predation demand (annual ration) of age-4+ pollock, Pacific cod,
and arrowtooth flounder in 2024 was 5.16 hundred thousand tons, below
the 6.17 hundred thousand ton annual average (Fig. 4).

\subsection{Implications:}\label{implications}

We find evidence of increased predation mortality on age-1 pollock,
Pacific cod, and arrowtooth flounder due to the species modelled in
CEATTLE. Previous ecosystem modelling efforts have estimated that
mortality of pollock is primarily driven by Pacific cod (16\%), Pacific
halibut (23\%) and arrowtooth flounder (33\%)(Gaichas et al., 2015).
Recent increases in predator biomass are contributing to the increase in
total consumption and therefore increased predation mortality. Between
1990 and 2010, relatively high natural mortality rates reflect patterns
in annual demand for prey from arrowtooth flounder, whose biomass peaked
during this time period.

\subsection{Description of index:}\label{description-of-index}

We report trends in age-1 natural mortality for walleye pollock
(\emph{Gadus chalcogrammus}), Pacific cod (\emph{Gadus macrocephalus})
and arrowtooth flounder (\emph{Atheresthes stomias}), from the Gulf of
Alaska (USA). Total natural mortality rates are based on model estimated
sex-specific, time- and age-invariant residual mortality (M1) and model
estimates of time- and age-varying predation mortality (M2) produced
from the multi-species statistical catch-at-age assessment model (known
as CEATTLE; Climate-Enhanced, Age-based model with Temperature-specific
Trophic Linkages and Energetics). The model is based, in part, on the
parameterization and data used for recent stock assessment models of
each species (see Adams et al., 2022 for more detail). The model is fit
to data from five fisheries and seven surveys between 1977 and 2024 and
includes inputs of abundance-at-age from recent stock assessment models
for Pacific halibut scaled to the proportion of age-5+ biomass in IPHC
management area 3 (Stewart \& Hicks, 2021). Model estimates of predation
mortality are empirically derived by bioenergetics-based consumption
information and diet data from the GOA to inform predator-prey
suitability (Holsman \& Aydin, 2015; Holsman, Aydin, Sullivan, Hurst, \&
Kruse, 2019).

\subsection{Literature Cited}\label{literature-cited}

Adams, G. D., Holsman, K. K., Barbeaux, S. J., Dorn, M. W., Ianelli, J.
N., Spies, I., Stewart, I. J., et al.~2022. An ensemble approach to
understand predation mortality for groundfish in the Gulf of Alaska.
Fisheries Research, 251: 106303.

Holsman, K. K., Ianelli, J., Aydin, K., Punt, A. E., and Moffitt, E. A.
2016. A comparison of fisheries biological reference points estimated
from temperature-specific multi-species and single-species
climate-enhanced stock assessment models. Deep Sea Research Part II:
Topical Studies in Oceanography, 134: 360--378.

Holsman, KK and K Aydin. (2015). Comparative methods for evaluating
climate change impacts on the foraging ecology of Alaskan groundfish.
Mar Ecol Prog Ser 521:217-23510.3354/ meps11102

Holsman, K.K., Aydin, K., Sullivan, J., Hurst, T., Kruse, G.H., 2019.
Climate effects and bottom-up controls on growth and size-at-age of
Pacific halibut (Hippoglossus stenolepis) in Alaska (USA). Fisheries
Oceanography, 28: 345--358. \url{doi:10.1111/fog.12416}

Gaichas, S., Aydin, K., and Francis, R. C. 2015. Wasp waist or beer
belly? Modeling food web structure and energetic control in Alaskan
marine ecosystems, with implications for fishing and environmental
forcing. Progress in Oceanography, 138: 1--17. Elsevier
Ltd.~\url{http://dx.doi.org/10.1016/j.pocean.2015.09.010}.

Stewart, I., Hicks, A., 2019. Assessment of the Pacific halibut
(\emph{Hippoglossus stenolepis}) stock at the end of 2018. International
Pacific Halibut Commission. Seattle, Wa, USA.

\newpage

\subsection{Figures:}\label{figures}

\begin{figure}
\centering
\includegraphics[width=0.8\textwidth,height=\textheight]{Results/ESR_Fig1.jpg}
\caption{Annual variation in natural mortality (\textbf{M1+M2}) of age-1
pollock (a), Pacific cod (b), and arrowtooth flounder (females and
males) (c/d) from the single-species models (dashed line), and the
multi-species models with temperature (points; solid line is a loess
polynomial smoother indicating trends over time)}
\end{figure}

\begin{figure}
\centering
\includegraphics[width=0.8\textwidth,height=\textheight]{Results/ESR_Fig2.jpg}
\caption{Multispecies estimates of biomass consumed as prey across all
ages by all predators annually in the model of walleye pollock (a),
Pacific cod (b), and arrowtooth flounder (c). Points represent annual
estimates, gray lines indicate 1979-2024 mean estimates for each
species, and the solid line is a 10 year (symmetric) loess polynomial
smoother indicating trends over time.}
\end{figure}

\begin{figure}
\centering
\includegraphics[width=0.8\textwidth,height=\textheight]{Results/ESR_Fig3.jpg}
\caption{Proportion of total predation mortality for age-1 pollock from
pollock (solid), Pacific cod (dashed), and arrowtooth flounder (dotted)
predators across years. Updated from Adams et al.~2022.}
\end{figure}

\begin{figure}
\centering
\includegraphics[width=0.8\textwidth,height=\textheight]{Results/ESR_Fig4.jpg}
\caption{Multispecies estimates of annual ration (hundred thousand tons
consumed per species per year) for adult (age 4 +) predators: pollock
(a), Pacific cod (b), and arrowtooth flounder (c). Gray lines indicate
1979 -2024 mean estimates and 1 SD for each species; solid line is a 10
y (symmetric) loess polynomial smoother indicating trends in ration over
time.}
\end{figure}

\end{document}
